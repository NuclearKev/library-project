\documentclass[12pt,a4paper,titlepage]{article}

\usepackage{times}
\usepackage{setspace}
\usepackage{textcomp}
\usepackage[toc,page]{appendix}
\usepackage{graphicx}
\usepackage{amsmath}
\usepackage{siunitx}
\usepackage[english]{babel}
\usepackage{biblatex}
\usepackage{csquotes}

\usepackage[bookmarks]{hyperref}
\hypersetup{colorlinks=true,allcolors=black}
\usepackage{hypcap}


\bibliography{sources.bib}

\doublespacing
\voffset = -25pt
\headsep = 35pt
\oddsidemargin = 0pt
\topmargin = 0pt
\marginparwidth = 0pt
\textwidth = 450pt
\textheight = 625pt

\begin{document}

\pagestyle{headings}
\setcounter{page}{1}

%---Title Page---%
\thispagestyle{empty}

\begin{flushleft}
\vspace*{1.5in}

{\huge The Enigma Machine}

\vspace{0.25in}

{\Large Kevin Bloom \\ Jonathan Covert}

\vspace{0.25in}

\vfill

\end{flushleft}

\newpage

\tableofcontents

\newpage

\section{Introduction}
This is a test~\cite{hold}.

\section{History}
The Enigma Machine was the Germans way of keeping their communications secure
during the early 1930s and beginning of World War 2. The Germans believe that
this machine's encryption could not be cracked because of the theoretical

\section{Operation}

\section{Something Else}

\newpage

\appendix

\section{Appendix: PSpice Waveforms}

\newpage
\printbibliography


\end{document}
