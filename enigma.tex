\documentclass[12pt,a4paper,titlepage]{article}

\usepackage{times}
\usepackage{setspace}
\usepackage{textcomp}
\usepackage[toc,page]{appendix}
\usepackage{graphicx}
\usepackage{amsmath}
\usepackage{siunitx}
\usepackage[english]{babel}
\usepackage{biblatex}
\usepackage{csquotes}

\usepackage[bookmarks]{hyperref}
\hypersetup{colorlinks=true,allcolors=black}
\usepackage{hypcap}


\bibliography{sources.bib}

\doublespacing
\voffset = -25pt
\headsep = 35pt
\oddsidemargin = 0pt
\topmargin = 0pt
\marginparwidth = 0pt
\textwidth = 450pt
\textheight = 675pt

\begin{document}

\pagestyle{headings}
\setcounter{page}{1}

%---Title Page---%
\thispagestyle{empty}

\begin{flushleft}
\vspace*{1.5in}

{\huge The Enigma Machine}

\vspace{0.25in}

{\Large Kevin Bloom \\ Jonathan Covert}

\vspace{0.25in}

\vfill

\end{flushleft}

\newpage

\tableofcontents

\newpage

\section{Introduction}
This is a test~\cite{hold}.

\section{The Enigma Machine}

\subsection{History}
The Enigma Machine was the Germans way of keeping their communications secure 
during the early 1930s and beginning of World War 2. The Germans believe that 
this machine's encryption could not be cracked because of the theoretical number 
of ciphering possibilities (approximately $3 \times 10^{114}$). The machine used 
a number of rotor wheels and a plugboard to obtain this insane number of 
possibilities. Around 1928, the Polish noticed that the Germans changed their 
communication encrypt and their normal deciphering techniques became useless. 
All their attempts at cracking the code were useless until they hired three 
mathematicians in 1932. These mathematicians eventually designed an equation 
that could determine the wiring connections on the current encryption 
settings.

Also involved in the uproar was Britain, who started hiring mathematicians to
try to crack the Enigma's code; one of those mathematicians was Alan Turing.
Turing was the man who is responsible for the demise of the Enigma. He design a
machine (called the Bombe) that could decrypt the Enigma's current code, giving
all the German's secret messages to the British. Because of this machine, the
outcome of World War 2 was greatly altered and many lives were saved. Turing has
been called ``the founder of computer science'', according to Andrew Hodges'
website about Turing, because of his development of the Bombe and of other
computer related subjects~\cite{wilcox2006solving}.

\subsection{Operation}
The Enigma Machine's operation was quite difficult and took a lot of skill to
fully understand. As mentioned earlier, the machine uses a bunch of rotors and a
plugboard to get the ``random'' encryption. Inside of the rotors were wires that
connected to each of the 26 input contacts. The wirings were different for each
rotor used. On the outside of the rotors there was a notch that could be moved
in order force the rotor to move a step forward. This notch was set differently
each day according to the key list that all German code clerks had.

The plugboard made things more complicated. All the plugboard did was swap two
letters with each other. The positions of the plugboard wires were, once again,
found on the key list and changed each day. That being said, each day, the
clerk would find the current date on the key list and set up the machine for
that day. They would plug in the plugboard wirings, select the rotors to be
used, change the notches, and place the rotors in the appropriate order. Before
any message was sent, the clerk would chose the three letters that appeared
in the window next to the rotors. This code was the initial rotor settings for
the current message. In order to decipher a message transmitted with an Enigma
code, the receiver needs to have all the same settings as the transmitter. This
is why all clerks had the same key list. The Enigma was most definitely not an
easy machine to understand, operator, or figure out. 


\section{Encryption Methods}
In modern times, there are many methods of encrypting data. The method that you
use will depend on what you are trying to accomplish. For example, if you are 
trying to do secure transfer of data between two sources that have never met, 
you'll want to use something with a private and public keys. If you're trying 
to encrypt data on your hard drive, private public keys don't even make sense. 

One method of encrypting is the Advanced Encryption Standard (AES)\@. It was 
created by the U.S. National Institute of Standards and Technology (NIST) in 
2001. Originally it was used for documents that contained ``sensitive but not 
classified'' information~\cite{daemen2013design}. AES was meant to replace DES 
which was used for banks, industry, and administration around the world. So it 
was clear that the AES was going to expand beyond its original intent. The AES 
is the only encryption standard approved by the NSA~\cite{daemen2013design}.

\section{Something Else}

\section{Global Collaboration}
\label{sec:global_collaboration}



\newpage

\appendix

\newpage
\printbibliography


\end{document}
