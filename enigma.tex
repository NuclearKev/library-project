\documentclass[12pt,a4paper,titlepage]{article}

\usepackage{times}
\usepackage{setspace}
\usepackage{textcomp}
\usepackage[toc,page]{appendix}
\usepackage{graphicx}
\usepackage{amsmath}
\usepackage{siunitx}
\usepackage[english]{babel}
\usepackage{biblatex}
\usepackage{csquotes}

\usepackage[bookmarks]{hyperref}
\hypersetup{colorlinks=true,allcolors=black}
\usepackage{hypcap}


\bibliography{sources.bib}

\doublespacing
\voffset = -25pt
\headsep = 35pt
\oddsidemargin = 0pt
\topmargin = 0pt
\marginparwidth = 0pt
\textwidth = 450pt
\textheight = 625pt

\begin{document}

\pagestyle{headings}
\setcounter{page}{1}

%---Title Page---%
\thispagestyle{empty}

\begin{flushleft}
\vspace*{1.5in}

{\huge The Enigma Machine}

\vspace{0.25in}

{\Large Kevin Bloom \\ Jonathan Covert}

\vspace{0.25in}

\vfill

\end{flushleft}

\newpage

\tableofcontents

\newpage

\section{Introduction}
This is a test~\cite{hold}.

\section{History}
The Enigma Machine was the Germans way of keeping their communications secure
during the early 1930s and beginning of World War 2. The Germans believe that
this machine's encryption could not be cracked because of the theoretical number
of ciphering possibilities (approximately $3 \times 10^{114}$). The machine used a
number of rotor wheels and a plugboard to obtain this insane number of
possibilities. Around 1928, the Polish noticed that the Germans changed their
communication encrypt and their normal ciphering techniques became useless. All
their attempts at cracking the code were useless until they hired three
mathematicians in 1932. These mathematicians eventually designed an equation that
could determine the the wiring connections on the current encryption settings.
Also involved in the uproar was Britain, whom started hiring mathematicians to
try and crack the Enigma's code. One of those mathematicians was Alan Turing.
Turing was the man who is responsible for the demise of the Enigma. He design a
machine (called the Bombe) that could decrypt the Enigma's current code, giving
all the German's secret messages to the British. Because of this machine, the
outcome of World War 2 was greatly altered and history was made. Turing has been
called ``the founder of computer science'', according to Andrew Hodges' website
about Turing, because of his development of the Bombe and of other computer
related subjects.
\cite{wilcox2006solving}

\section{Encryption Methods}
In modern times, there are many methods of encrypting data. The method that you
use will depend on what you are trying to accomplish. For example, if you are 
trying to do secure transfer of data between two sources that have never met, 
you'll want to use something with a private and public keys. If you're trying 
to encrypt data on your hard drive, private public keys don't even make sense. 

One method of encrypting is the Advanced Encryption Standard (AES)\@. It was 
created by the U.S. National Institute of Standards and Technology (NIST) in 
2001. Originally it was used for documents that contained ``sensitive but not 
classified'' information~\cite{daemen2013design}. AES was meant to replace DES 
which was used for banks, industry, and administration around the world. So it 
was clear that the AES was going to expand beyond its original intent. The AES 
is the only encryption standard approved by the NSA~\cite{daemen2013design}.

\section{Something Else}

\section{Global Collaboration}
\label{sec:global_collaboration}



\newpage

\appendix

\newpage
\printbibliography


\end{document}
