\documentclass[12pt,a4paper,titlepage]{article}

\usepackage{times}
\usepackage{setspace}
\usepackage{textcomp}
\usepackage[toc,page]{appendix}
\usepackage{graphicx}
\usepackage{amsmath}
\usepackage{siunitx}

\doublespacing
\voffset = -25pt
\headsep = 35pt
\oddsidemargin = 0pt
\topmargin = 0pt
\marginparwidth = 0pt
\textwidth = 450pt
\textheight = 625pt

\begin{document}

\pagestyle{headings}
\setcounter{page}{1}

%---Title Page---%
\thispagestyle{empty}

\begin{flushleft}
\vspace*{1.5in}

{\huge LRC Circuit Report}

\vspace{0.25in}

{\Large Kevin Bloom}

\vspace{0.25in}

\vfill

\end{flushleft}

\newpage

\tableofcontents

\newpage

\section{Objectives}
There were a handful of objectives for this lab. Each either refreshing our
memories with old material or applying newer concepts that we have yet to
master. The first objective was a throwback to our Freshman days with Karl.
We needed to calculate the impedances, current, and voltages of all the
components in the circuit. With these values ready for comparison, we used
PSpice to model the circuit. The next objective was to build the circuit and
measure values with a DMM (to compare with the calculated). Lastly, 
we attempted to better our results by redoing the calculations with the actual
measured values of the components. 


\section{Calculations}
In this section, we will go through the calculations that I did. Most of this
section will be copied straight from my hand written copy just typed so it
looks nice and is easy to read.

\subsection{Nominal}
To start, we must first find all the impedances in the circuit. We start by finding
the capacitance and inductive reactance of the cap and inductor.
\begin{align*}
  X_C &= \frac{-j}{2\pi f C} \\
  &= \frac{-j}{2\pi (1000) (.033\si{\micro \farad})} \\
  &= -j4.823\si{\kilo \ohm}
\end{align*}
\begin{align*}
  X_L &= j2\pi f L \\
  &= j2\pi (1000) (940\si{\milli \henry}) \\
  &= j5.906\si{\kilo \ohm}
\end{align*}
Next we simply throw these values into the impedance form (including the resistor).
\begin{align*}
  Z_C &= 0 - j4.823\si{\kilo \ohm} \\
  &= 4.823\si{\kilo \ohm} \angle {-90}^{\circ} \\
  Z_L &= 0 + j5.906\si{\kilo \ohm} \\
  &= 5.906\si{\kilo \ohm} \angle 90^{\circ} \\
  Z_R &= 2.4\si{\kilo \ohm} \angle 0^{\circ}
\end{align*}
Now that we have all the impedances we can add them up to get the total and divide
by the total voltage (our source) to get us the current of this circuit.
\begin{align*}
  Z_T &= Z_C + Z_L + Z_R \\
  &= 4.823\si{\kilo \ohm} \angle {-90}^{\circ} + 5.906\si{\kilo \ohm} \angle 90^{\circ} + 2.4\si{\kilo \ohm} \angle 0^{\circ} \\
  &= 2.633\si{\kilo \ohm} \angle 24.294^{\circ} \\
  I_\text{AC} &= \frac{\si{\volt}_\text{AC}}{Z_T} \\
  &= \frac{\frac{3.4\sqrt{2}}{2}\si{\volt}_\text{AC}}{2.633\si{\kilo \ohm} \angle 24.294^{\circ}} \\
  &= 913.03\si{\micro \ampere}_\text{AC} \angle {-24.294}^{\circ}
\end{align*}
Now that we have successfully calculated the series current, we can use Ohm's Law
to bust out the voltage drops across those components.
\begin{align*}
  V_C &= I_\text{AC}Z_C \\
  &= (913.03\si{\micro \ampere}_\text{AC} \angle {-24.294}^{\circ})(4.823\si{\kilo \ohm} \angle {-90}^{\circ}) \\
  &= 4.403\si{\volt}_\text{AC} \angle {-114.294}^{\circ} \\
  V_L &= I_\text{AC}Z_L \\
  &= (913.03\si{\micro \ampere}_\text{AC} \angle {-24.294}^{\circ})(5.906\si{\kilo \ohm} \angle 90^{\circ}) \\
  &= 5.393\si{\volt}_\text{AC} \angle 65.706^{\circ} \\
  V_R &= I_\text{AC}Z_R \\
  &= (913.03\si{\micro \ampere}_\text{AC} \angle {-24.294}^{\circ})(2.4\si{\kilo \ohm} \angle 0^{\circ}) \\
  &= 2.191\si{\volt}_\text{AC} \angle {-24.294}^{\circ}
\end{align*}
Lastly, to find the voltage a point Q with respect to ground, we can add the voltage drops across the resistor and the inductor.
\begin{align*}
  V_Q &= V_R + V_L \\
  &= 5.393\si{\volt}_\text{AC} \angle 65.706^{\circ} + 2.191\si{\volt}_\text{AC} \angle {-24.294}^{\circ} \\
  &= 5.821\si{\volt}_\text{AC} \angle 43.592^{\circ}
\end{align*}


\subsection{Actual}
In order to get a better model of the circuit, I recalculated all the values
but with the actual measured values of the components, rather than the nominal
ones. The values that I measured from each components are as listed below.
\begin{align*}
  C &= .035\si{\micro \farad} \\
  L &= 950\si{\milli \henry} \\
  R_L &= 368.76\Omega \\
  R &= 2.396\si{\kilo \ohm}
\end{align*}
If we use these values instead of the nominal values we should get values that
more closely resemble our measured values. I did the same exact work and ended
up with the following values.
\begin{align*}
  Z_C &= 4.547\si{\kilo \ohm} \angle {-90}^{\circ} \\
  Z_L &= 5.980\si{\kilo \ohm} \angle 86.465^{\circ} \\
  Z_R &= 2.396\si{\kilo \ohm} \angle 0^{\circ} \\
  Z_T &= 3.109\si{\kilo \ohm} \angle 27.214^{\circ} \\ 
  I_\text{AC} &= 773.317\si{\micro \ampere}_\text{AC} \angle {-27.214}^{\circ} \\ 
  V_C &= 3.516\si{\volt}_\text{AC} \angle {-117.214}^{\circ} \\
  V_L &= 4.626\si{\volt}_\text{AC} \angle 59.251^{\circ} \\ 
  V_R &= 1.853\si{\volt}_\text{AC} \angle {-27.214}^{\circ} \\
  V_Q &= 5.087\si{\volt}_\text{AC} \angle 37.933^{\circ}
\end{align*}


\section{Comparison Between Calculations and Measured Values}

\subsection{Numerical Analysis}
Before we start talking about the discrepancies between the theoretical and
measured values, I should probably actually display my measured values.
\begin{align*}
  V_C &= 2.912 \si{\volt}_\text{AC} \\
  V_L &= 3.994 \si{\volt}_\text{AC} \\
  V_R &= 1.544 \si{\volt}_\text{AC} \\
  V_Q &= 5.538 \si{\volt}_\text{AC}
\end{align*}

If we compare these values to what we got when we used nominal values to
calculate theoretical values you will notice something isn't right. $V_C$ is off
by $1.491\si{\volt}_\text{AC}$, $V_L$ is off by $1.399\si{\volt}_\text{AC}$, and $V_R$ is off by $647\si{\milli \volt}_\text{AC}$.
These are percent differences of $33.863\%$, $25.941\%$, and $29.53\%$,
respectively. These are huge errors! What is going on here!? Well, before we
get our underwear in a bunch lets look at the actual calculations and seen if
they're better. However, what you will see is that they aren't much better.
With percent errors of $17.179\%$, $13.662\%$, and $16.676\%$, respectively.
This is still a huge amount of error; lets see if we can figure out what might
be causing this error.

Well first and foremost, there is a small amount of error from the get go. Was
the function generator set exactly to $1\si{\kilo \hertz}$? Can the function generator even
output an exact wave? Was the amplitude exactly $3.4\si{\volt}_{P}$? The answer to these
questions are obviously no. However, this would only cause a slight error,
because we are talking being a few Hertz or a few millivolts off. Also, our
cap and resistor seemed to be pretty close; the resistor only being off by about
$0.06\%$. However, I believe the culprit for our skewed data, is the inductor.
Why you ask? Well, since there is no such thing as a $940\si{\milli \henry}$ inductor, we had
to construct our own equivalent inductance value by using series inductors.
When we measured the inductance of this series inductor chain, it bounced around
$940\si{\milli \henry}$ to $960\si{\milli \henry}$. With this big of a tolerance, we could see a huge error in
out measured data. Not to mention that when we use the nominal value for the
inductor we miss out on the winding resistance. As you can see in the actual
calculations, they are much closer than when we used the nominal.
This is why there is such a huge error when comparing our theoretical data to
our measured data.

\subsection{Waveform Analysis}
Now that we have figured out where our error is coming from when comparing the
numbers, lets take a look at our waveforms and see how they compare. You can see
the waveforms in the Appendix (PSpice in Appendix A and measured in Appendix B).
According to Appendix A, $V_Q$ (A.1) has a phase of about $48^{\circ}$ and $V_L$ (A.2) has
a phase of $72^{\circ}$. When compared to our nominal calculated phases, they are off by
about $11^{\circ}$ and $13^{\circ}$, respectively. According to Appendix A.2,
they have very similar phases, possibly being off by a few decimals. However, if
we compare our actual component value calculations to those phases, we get less
error! They are only off by about $5^{\circ}$ and $7^{\circ}$, respectively.

Lets take a look at our measured waveforms and see how they compare to our
calculated and PSpice waveforms. According to Appendix B, $V_Q$ (B.1) has a phase
of about $36^{\circ}$ and $V_L$ (B.2) has a phase of about $72^{\circ}$. Comparing
these values to our nominal calculations we see a difference of about $7^{\circ}$
for both phases. If we look at our actual component value calculations, we see
a difference of about $1^{\circ}$ (best one yet) and $13^{\circ}$, respectively.

Once again we are seeing large errors in our data. I believe that the inductors
are the culprit in this case as well. In fact, we can almost for sure say it's the
inductors because of the error in the phase angles.


\section{Conclusion}
In this section we will tie up some loose ends and answers a few possible
questions we might pose in during or after the lab.

\subsection{Nominal vs Actual in Calculations}
As we saw in the numerical and waveform analysis sections, we generally saw
that our actual calculations contained less error than the nominal ones. What
this means is that it is better use the actual measured values of your
components than using the nominal values in a schematic. If the planets align
in the right way, you should see a dramatic decrease in error from doing this.
In our case, the planets didn't align all the way and we still saw a lot of
error in our data. Luckily, we have narrowed it down to the inductor chain
that was causing this nonsense. 

\subsection{Tolerances of Measuring Instruments}
I know we blamed most of the error on the inductors but there is still a minuscule
amount of error coming from the test equipment itself. In the numerical analysis
we talked about how the function generator could have been causing small amounts
of error, but what about the DMM or oscilloscope? When taking measurements with
a DMM, there is also extra resistance from the leads. This can easily skew
resistance data or even voltage/current data as well. Not to mention that the
device itself can only be so accurate. The DMM might round a couple decimals
or just flat out not be able to read accurately enough to get a close to perfect
measurement. The oscilloscope on the other hand has a larger amount of error.
Mainly because it relies on the operator to read (as accurately as he/she can)
the values off the screen rather than spitting out an ``exact'' number. As you
can see, we can't ignore these errors, even though they are small.

\subsection{Proving KVL using DMM values}
In theory, we should be able to add up all the voltage drops across the
components and end up with the magnitude of the source. However, if we try this
with the measured values that I took, you'll see this isn't true. Why doesn't
this work? Did Kirchoff lie to me?! No, he didn't. The reason this doesn't
work is because of the phase shifts. We must include the phases in our
calculation if we wish to see KVL actually work. If you attempt to ball park
the phases for the measured voltages, you'll see that you get close to the 
source value. If we were to have the exact phases for
these values, we would see the KVL works perfectly. However, this is not the
case.

\subsection{Versatility of Instruments}
Throughout this lab, we switched between a DMM and oscilloscope a couple of times.
This is a good example on how these instruments are very versatile; mainly the
DMM. We can use the DMM to measure voltage (DC and AC), current (DC and AC),
resistance (2-wire and 4-wire), frequency, and much more. These high quality
DMMs can really do it all. The oscilloscope on the other hand is a little more
versatile but still has cons. We can measure just about everything the DMM
can but we lack the accuracy. Also, the oscilloscope has an added bonus,
namely, waveforms. We can actually see what the wave looks like right on the
screen. Each of these pieces of equipment have their pros and cons, which is
why it's good to have both and pick which one you want to use based on the
problem at hand.

\newpage

\appendix

\section{Appendix: PSpice Waveforms}

\subsection{Nominal Component Values}
\begin{figure}[ht!]
\centering
\includegraphics[width=150mm]{/Users/kevin/Desktop/pspice-waveform.png}
\caption{Green: $V_S$, Red: $V_Q$, Blue: $V_L$ \label{overflow}}
\end{figure}

\subsection{Actual Component Values}
\begin{figure}[ht!]
\centering
\includegraphics[width=150mm]{/Users/kevin/Desktop/actual-values-waveform.png}
\caption{Green: $V_S$, Red: $V_Q$, Blue: $V_L$ \label{overflow}}
\end{figure}

\section{Appendix: Oscilloscope Waveforms}

\subsection{$V_L$ with $V_S$}
\begin{figure}[ht!]
\centering
\includegraphics[width=90mm]{/Users/kevin/Downloads/IMG_0792.JPG}
\caption{Yellow: $V_S$, Green: $V_Q$ \label{overflow}}
\end{figure}


\subsection{$V_Q$ with $V_S$}
\begin{figure}[ht!]
\centering
\includegraphics[width=90mm]{/Users/kevin/Downloads/IMG_0793.JPG}
\caption{Yellow: $V_S$, Green: $V_L$ \label{overflow}}
\end{figure}


\end{document}
